\documentclass{article}
\usepackage[margin=1in]{geometry}
\usepackage{amsmath}
\usepackage{amssymb}
\usepackage[colorlinks=true]{hyperref}

\begin{document}
\begin{center}
  {\bfseries Assignment \#2}\\
  Introduction to GR, 2020 Fall\\
  International Centre for Theoretical Sciences\\
  Instructor: Prof. Bala Iyer, Tutor: Md Arif Shaikh\footnote{\href{mailto: arif.shaikh@icts.res.in}{arif.shaikh@icts.res.in}}\\
  Due on October 04, 2020 11:59 PM.
\end{center}
\hrule

\begin{enumerate}
\item If $s$ is an affine parameter, show that, under the transformation $s\to \bar{s} = \bar{s}(s)$, the parameter $\bar{s}$ will be affine only if $\bar{s} = \alpha s + \beta$, where $\alpha$ and $\beta$ are constants.

\item Show that if $t^\alpha = dx^\alpha/d\lambda$ obeys the geodesic equation in the form $Dt^\alpha/d\lambda = kt^\alpha$, then $u^\alpha = dx^\alpha/d\lambda^\star$ satisfies $Du^\alpha/d\lambda^\star = 0$ if $\lambda^\star$ and $\lambda$ are related by $d\lambda^\star/d\lambda = \exp(\int k (\lambda) d\lambda)$.

\item Prove that if a manifold is affine  flat then the connection is necessarily integrable and symmetric.

\item Find the geodesic equation for $\mathbb{R}^3$ in cylindrical polars.

\item Prove that the covariant derivative of the metric tensor vanishes.

\item Establish the theorem that any two-dimensional Riemann  manifold is conformally flat in the case of a metric of signature 0, i.e., at any point the metric can be reduced to the diagonal form $(+1 -1 )$ say.

\item Show that the Einstein tensor $G_{ab}$ vanishes if and only if the Ricci tensor $R_{ab}$ vanishes.

\item Let's take the coordinates $x^a = (t, r, \theta, \phi)$ and the line element
  \begin{equation}
    \label{eq:line-element}
    ds^2 = e^\nu dt^2 - e^\lambda dr^2 - r^2d\theta^2 - r^2\sin^2\theta d\phi^2,
  \end{equation}
  where $\nu = \nu(t, r)$ and $\lambda = \lambda(t, r)$ are arbitrary functions of $t$ and $r$.
  \begin{enumerate}
  \item Find $g_{ab}$, $g$ and $g^{ab}$.
  \item Calculate the connections using the metric elements.
  \item Calculate the Riemann tensor.
  \item Calculate the Ricci tensor, Ricci scalar and the Einstein tensor.
  \end{enumerate}

\item The surface of a two-dimensional cone is embedded in three-dimensional flat space. The cone has an opening angle of $2\alpha$. Points on the cone which all have the same distance $r$ from the apex define a circle, and $\phi$ is the angle that runs along the circle.
  \begin{enumerate}
  \item Write down the metric of the cone in terms of $r$ and $\phi$.
  \item Find the coordinate transformations $x(r, \phi)$ and $y(r, \phi)$ that brings the metric into the form $ds^2 = dx^2 + dy^2$. Do these coordinates cover the entire two-dimensional plane?
  \item Prove that any vector parallel transported along a circle of constant $r$ on the surface of the cone ends up rotated by an angle $\beta$ after a complete trip. Express $\beta$ in terms of $\alpha$.
  \end{enumerate}
\end{enumerate}
\end{document}

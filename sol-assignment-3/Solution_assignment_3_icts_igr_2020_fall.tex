\documentclass[fleqn]{article}
\usepackage[margin=1in]{geometry}
\usepackage{amsmath}
\usepackage{amssymb}
\usepackage[colorlinks=true]{hyperref}
\usepackage[makeroom]{cancel}

\begin{document}
\begin{center}
  {\bfseries Solution to assignment \#3}\\
  Introduction to GR, 2020 Fall\\
  International Centre for Theoretical Sciences\\
  Instructor: Prof. Bala Iyer, Tutor: Md Arif Shaikh\footnote{\href{mailto: arif.shaikh@icts.res.in}{arif.shaikh@icts.res.in}}\\
\end{center}
\hrule

\begin{enumerate}
\item[3.] Prove that
  \begin{equation}
    \label{eq:Killing-vectors-sph}
    \xi^{1} = \sin\phi\partial_\theta + \cot\theta\cos\phi\partial_\phi\qquad \xi^{2} = - \cos\phi\partial_\theta + \cot\theta\sin\phi\partial_\phi
  \end{equation}
  are the Killing vectors of the spherically symmetric spacetime
  \begin{equation}
    \label{eq:line-element-spherically-symmetric}
    ds^2 = -A(r)dt^2 + B(r)dr^2 + r^2 (d\theta^2 + \sin^2\theta d\phi^2)
  \end{equation}
  \noindent {\bfseries Answer:} Killing vectors satisfy the Killing equation
  \begin{equation}
    \label{eq:Killing-equation}
    \nabla_\alpha \xi_\beta + \nabla_\beta \xi_\alpha = 0.
  \end{equation}
  Thus we need to show that $\nabla_\alpha\xi_\beta$ is an anti-symmetric tensor. We do it for $\xi^1$. The contravariant components are
  \begin{equation}
    \label{eq:contravariant-xi1}
    \xi^1_t = \xi^1_r = 0,\quad \xi^1_\theta = g_{\theta\theta}\xi^{1\theta} = r^2\sin\phi,\quad\xi^1_{\phi} = g_{\phi\phi}\xi^{1\phi} = r^2\sin^2\theta \cot\theta\cos\phi = r^2\sin\theta\cos\theta\cos\phi.
  \end{equation}
  Since only $\xi_\theta$ and $\xi_\phi$ are non-vanishing, the relevant Christofell symbols are $\Gamma^\theta_{\alpha\beta}$ and $\Gamma^\phi_{\alpha\beta}$. These we get from the geodesic equation for $\theta$ and $\phi$ using $\mathcal{L}^2 = (1/2)g_{\alpha\beta}\dot{x}^\alpha\dot{x}^\beta$ as the Lagrangian. For $\theta$, we have
  \begin{equation}
    \label{eq:theta-e-o-m}
    \ddot{\theta} + \frac{2}{r}\dot{r}\dot{\theta} - \sin\theta\cos\theta\ddot{\phi} = 0,
  \end{equation}
  which provides
  \begin{equation}
    \label{eq:Gamma-theta}
    \Gamma^\theta_{r\theta} = \frac{1}{r},\quad\Gamma^\theta_{\phi\phi} = - \sin\theta\cos\theta.
  \end{equation}
  Similarly for $\phi$,
  \begin{equation}
    \label{eq:phi-e-o-m}
    \ddot{\phi} + \frac{2}{r}\dot{r}\dot{\phi}+2\cot\theta\dot{\theta}\dot{\phi} = 0,
  \end{equation}
  which provides
  \begin{equation}
    \label{eq:Gamma-phi}
    \Gamma^\phi_{r\phi}= \frac{1}{r},\quad\Gamma^\phi_{\theta\phi} = \cot\theta.
  \end{equation}
  Now that we have the required Christofell symbols, we check that $\nabla_\alpha\xi^1_\beta$ is an anti-symmetric tensor.
  \begin{align*}
    \boxed{\nabla_t\xi^1_t = \nabla_r\xi^1_r = 0}
  \end{align*}
  trivially. and
  \begin{align*}
    \boxed{\nabla_\theta\xi^1_\theta = 0,\quad \nabla_\phi\xi^1_\phi = \partial_\phi\xi^1_\phi - \Gamma^\phi_{\phi\phi} = 0}
  \end{align*}
  $\nabla_t \xi^1_r = - \nabla_r\xi^1_t = 0$. $\nabla_t\xi^1_\theta = 0 = - \nabla_\theta\xi^1_t$. $\nabla_t\xi^1_\phi = 0 = - \nabla_\phi\xi^1_t$.
  \begin{align*}
    \nabla_r\xi^1_\theta
    & = \partial_r\xi^1_\theta - \Gamma^{\theta}_{r\theta}\xi^1_\theta\\
    & = 2r\sin\phi - \frac{1}{r}r^2 \sin\phi = r\sin\phi.
  \end{align*}
  \begin{align*}
    \boxed{\nabla_\theta\xi^1_r = - \Gamma^\theta_{r\theta}\xi^1_\theta = - r\sin\phi = - \nabla_r\xi^1_\theta}.
  \end{align*}
  \begin{align*}
    \nabla_r\xi^1_\phi
    & = \partial_r\xi^1_\phi - \Gamma^\phi_{r\phi}\xi^1_\phi\\
    & = 2r\sin\theta\cos\theta\cos\phi - \frac{1}{r} r^2\sin\theta\cos\theta\cos\phi\\
    & = r\sin\theta\cos\theta\cos\phi.
  \end{align*}
  \begin{align*}
    \boxed{\nabla_\phi\xi^1_r = - \Gamma^\phi_{r\phi}\xi^1_\phi = - r\sin\theta\cos\theta\cos\phi = - \nabla_r\xi^1_\phi}.
  \end{align*}
  \begin{align*}
    \nabla_\theta\xi^1_\phi
    & = \partial_\theta\xi^1_\phi - \Gamma^\phi_{\theta\phi}\xi^1_\phi\\
    & = r^2\cos 2\theta\cos\phi - \cot\theta r^2\sin\theta\cos\theta\cos\phi.\\
    & = r^2\cos\phi(\cos2\theta - \cos^2\theta)\\
    & = r^2\cos\phi(-\sin^2\theta)
  \end{align*}
  \begin{align*}
    \nabla_\phi\xi^1_\theta
    & = \partial_\phi\xi^1_\theta - \Gamma_{\theta\phi}^\phi\xi^1_\phi\\
    & = r^2\cos\phi - \cot\theta r^2\sin\theta\cos\theta\cos\phi\\
    & = r^2\cos\phi(1 - \cos^2\theta)\\
    & = r^2\cos\phi\sin^2\theta\\
    & \Rightarrow\boxed{\nabla_\phi\xi^1_\theta = - \nabla_\theta\xi^1_\phi}
  \end{align*}

  Similarly, we can check for $\xi^2$.
  
\item[4.] A particle with electric charge $e$ moves in a spacetime with metric $g_{\alpha\beta}$ in the presence of a vector potential $A_\alpha$. The equations of motion are $u^\beta \nabla_\beta u_\alpha = e F_{\alpha\beta}u^\beta$, where $u^\alpha$ is the four-velocity and $F_{\alpha\beta}=\nabla_\alpha A_\beta - \nabla_\beta A_\alpha$. It is assumed that the spacetime possesses a Killing vector $\xi^\alpha$, so that $\pounds_\xi g_{\alpha\beta} = \pounds_\xi A_\alpha = 0$. Prove that
  \begin{equation}
    \label{eq:conserved-quantity-em-field}
    (u_\alpha + e A_\alpha)\xi^\alpha
  \end{equation}
  is constant on the world line of the of the charged particle.

  \noindent {\bfseries Answer:}
  \begin{align*}
    \frac{d}{d\lambda}\left[(u_\alpha + e A_\alpha)\xi^\alpha\right]
    & = u^\beta\nabla_\beta\left[(u_\alpha + e A_\alpha)\xi^\alpha\right]\\
    & = u^\beta(\nabla_\beta u^\alpha)\xi^\alpha + \underbrace{u^\beta u^\alpha}_{\text{symmetric}}\times \underbrace{\nabla_\beta \xi^\alpha}_{\text{anti-symmetric}} + e \xi^\alpha u^\beta\nabla_\beta A_\alpha + e A_\alpha u^\beta \nabla_\beta \xi^\alpha\\
    & = eu^\beta(\nabla_\alpha A_\beta - \cancel{\nabla_\beta A_\alpha})\xi^\alpha + \cancel{e\xi^\alpha u^\beta \nabla_\beta A_\alpha} + e A_\alpha u^\beta \nabla_\beta \xi^\alpha\\
    & = e u^\beta (\xi^\alpha\nabla_\alpha A_\beta + A_\alpha \nabla_\beta \xi^\alpha)\\
    & = e u^\beta \pounds_\xi A_\alpha\\
    & = 0
  \end{align*}

\item[5.] A particle moving on a circular orbit in a stationary, axially symmetric spacetime is subjected to a dissipative force which drives it to another, slightly smaller, circular orbit. During the transition, the particle loses an amount $\delta \tilde{E}$ of orbital energy (per unit rest mass) and an amount $\delta \tilde{L}$ of orbital angular momentum (per unit rest mass). Show that these quantities are related by $\delta\tilde{E} = \Omega \delta\tilde{L}$, where $\Omega$ is the particle's original angular velocity.
  
  {\itshape Hints:} Express the four-velocity $u^\alpha$ of the particle in terms of the Killing vectors, energy angular momentum and orbital velocity. Find the variation $\delta u^\alpha$. Use the normalization condition $u_\alpha u^\alpha = -1$.

  \noindent {\bfseries Answer:} The spacetime possesses the Killing vectors $t^\alpha = \delta^\alpha_t$ and $\phi^\alpha = \delta^\alpha_\phi$. In terms of these we can write down the four-velocity $u^\alpha$ as
  \begin{equation}
    \label{eq:four-velocity}
    u^\alpha = u^t t^\alpha + u^\phi \phi^\alpha = u^t(t^\alpha + \Omega \phi^\alpha),\qquad \Omega = \frac{u^\phi}{u^t}
  \end{equation}
  as the particle moves in a circular orbit. Using the normalization condition $u_\alpha u^\alpha = -1$ we find
  \begin{equation}
    \label{eq:normalization-condition}
    u_\alpha u^\alpha = u_tu^t + u_\phi u^t\Omega = -1
  \end{equation}
  or,
  \begin{equation}
    \label{eq:normalization-condition-2}
    u^t(u_t + u_\phi \Omega) = -1
  \end{equation}
  Now the two constants of motion from the two Killing vectors are $E = - t^\alpha u_\alpha = -u_t$ and $L = \phi^\alpha u_\alpha = u_\phi$. Thus in terms of $E$ and $L$, $u^t$ becomes
  \begin{equation}
    \label{eq:u^t}
    u^t = \frac{1}{E - \Omega L}
  \end{equation}
  and hence the four-velocity could be written as
  \begin{equation}
    \label{eq:four-velocity-2}
    u^\alpha = \frac{(t^\alpha + \Omega \phi^\alpha)}{(E -\Omega L)}
  \end{equation}
  The variation of the four-velocity would be given by
  \begin{equation}
    \label{eq:variation-u^alpha}
    \delta u^\alpha = -\frac{(t^\alpha + \Omega \phi^\alpha)}{(E -\Omega L)^2}(\delta E - \Omega \delta L) = - u^\alpha u^t (\delta E - \Omega \delta L)
  \end{equation}
  contracting the above with $u_\alpha$ gives
  \begin{equation}
    \label{eq:u_alpha-delta-u^alpha}
    u_\alpha \delta u^\alpha = u^t(\delta E - \Omega \delta L)
  \end{equation}
  Now, since $u_\alpha u^\alpha = -1$, variation of it should vanish which implies $u_\alpha \delta u^\alpha = 0$. This provides the relation
  \begin{equation}
    \label{eq:deltaE-deltaL}
    \boxed{\delta E = \Omega \delta L}.
  \end{equation}
\end{enumerate}
\end{document}
